%% start of file 'template.tex'.
%% Copyright 2006-2013 Xavier Danaux (xdanaux@gmail.com).
%
% This work may be distributed and/or modified under the
% conditions of the LaTeX Project Public License version 1.3c,
% available at http://www.latex-project.org/lppl/.



\documentclass[letterpaper]{moderncv}        % possible options include font size ('10pt', '11pt' and '12pt'), paper size ('a4paper', 'letterpaper', 'a5paper', 'legalpaper', 'executivepaper' and 'landscape') and font family ('sans' and 'roman')
\usepackage{textcomp}
\usepackage{comment}
% moderncv themes
\moderncvstyle{classic}                             % style options are 'casual' (default), 'classic', 'oldstyle' and 'banking'
\moderncvcolor{blue}                               % color options 'blue' (default), 'orange', 'green', 'red', 'purple', 'grey' and 'black'
%\renewcommand{\familydefault}{\sfdefault}         % to set the default font; use '\sfdefault' for the default sans serif font, '\rmdefault' for the default roman one, or any tex font name
%\nopagenumbers{}                                  % uncomment to suppress automatic page numbering for CVs longer than one page

% character encoding
\usepackage[utf8]{inputenc}                       % if you are not using xelatex ou lualatex, replace by the encoding you are using
%\usepackage{CJKutf8}                              % if you need to use CJK to typeset your resume in Chinese, Japanese or Korean

% adjust the page margins
\usepackage[scale=0.90]{geometry}
%\setlength{\hintscolumnwidth}{1.98cm}                % if you want to change the width of the column with the dates
%\setlength{\makecvtitlenamewidth}{10cm}           % for the 'classic' style, if you want to force the width allocated to your name and avoid line breaks. be careful though, the length is normally calculated to avoid any overlap with your personal info; use this at your own typographical risks...

    % Profile
\name{Artem Kulakevich}{}
\address{Beaverton, Oregon}
\phone[mobile]{503-750-3225}
\email{Artem3@pdx.edu}
\homepage{linkedin.com/in/artem-kulakevich/}
    \begin{document}
\makecvtitle
\vspace*{-.8cm}
    
\section{\underline{Education}}

\cventry
{Jun 2019 -- Present}
{Master of Science, Electrical Engineering}
{Portland State University}
{\textit{Portland, OR}}
{}
{\textbf{GPA: 4.00}, \textit{Expected: Jun 2021}}
{}
\cventry
{Sep 2017 -- Jun 2020}
{Bachelor of Science, Electrical Engineering}
{Portland State University}
{\textit{Portland, OR}}
{}
{\textbf{GPA: 3.97}, \textit{Summa cum laude}}
{}

\section{\underline{Work Experience}}
\cventry
{Dec 2016 -- Present}
{Production Specialist III}
{Micro Systems Engineering Inc.}
{Lake Oswego, OR}
{}
{\begin{itemize}%
		\item Certified for all back-end production processes; more certifications than anyone in area.
		\item Team lead for back-end tasks, designated to assign work, and resolve communication issues.
		\item Introduce new production processes, workflow changes, and training through various engineering projects.
		\item Work on LabVIEW software changes for production imaging cell, including documentation and code reviews.
	\end{itemize}}
	
\cventry
{Jun 2015 -- Dec 2016}
{Crew Member}
{Wendy's Restaurant}
{Portland OR}
{}
{\begin{itemize}%
		\item First job after high school to pay for college. Strong focus on communication and teamwork.
	\end{itemize}}

\section{\underline{Skills}}
\cvitem{\textbf{Languages}}{C++, C, Rust, LabVIEW 12.0, ARM Assembly, Matlab, SystemVerilog}
\cvitem{\textbf{Programs}}{Git, Linux (Ubuntu), Windows, LTspice, Cadence Virtuoso, Visual Studio, MS Office, SAP}
\cvitem{\textbf{Hardware}}{Soldering, Oscilloscope (Tektronix/Rigol), Function Generator (Tektronix), Power Supply}


\section{\underline{Projects}}
\cventry
{Jan 2020 -- Jun 2020}
{Senior Capstone | Galois Inc.}
{}
{\textit{Rust, C++, Arduino, Kind2, Lustre, PHP, SQL, Apache2}}
{}
{Modified Kind2 Lustre to Rust compiler to generate embedded Rust code from a verifiable language.  Streamlined the process of creating verifiable embedded controllers.  Found Rust PID controller to have identical real-world performance to controller written in C++.  Wrote and tested a Fuzzy control on a real-world system.\\}
	
\cventry
{Oct 2019 -- Jan 2020}
{Module Imaging Cell | MSEI}
{}
{\textit{LabVIEW 12.0, log4net}}
{}
{Implemented software changes to automated production cell software.  Introduced log4net logging to SQL, data collection to a digital factory, and changes to the state machine that reduced chances of collision and product loss.\\}

\cventry
{Apr 2020 -- Jun 2020}
{MIPS-lite Simulator | ECE 586}
{}
{\textit{C++, Git}}
{}
{Designed a 5-stage MIPS simulator in C++ and tested output with generic memory image provided by professor.\\}

\cventry
{Sep 2019 -- Feb 2020}
{CMOS Standard Library Design | ECE 526}
{}
{\textit{Virtuoso 6.1.8, ADE, OCEAN/SKILL}}
{}
{Designed standard library components using Cadence Virtuoso layout and ADE tools.  Wrote scripts to simulate and measure output values with different temperatures, inputs voltages, and input rise times.\\}	

\cventry
{Sep 2019 -- Dec 2020}
{Class AB Audio Amplifier | ECE 521}
{}
{\textit{LTspice, Soldering, Oscilloscope}}
{}
{Designed a complimentary symmetry audio amplifier using mostly discrete BJTs to drive a 10W speaker.  Soldered, designed and tested using homelab equipment.\\}	

\cventry
{Sep 2019}
{Interactive IMU Cube | ECE 411}
{}
{\textit{C++, Soldering, MS Project, Git}}
{}
{}	

%\cventry
%{Sep 2019 -- Dec 2020}
%{Interactive IMU Cube | ECE 411}
%{}
%{\textit{C++, Soldering, MS Project, Git}}
%{}
%{Worked with a team to design a PCB for ATmega328p with necessary bypass, addressable LEDs, and IMU. Personally programmed %processor to control LEDs using physical movement.\\}
	
%\cventry
%{Sep 2019}
%{Class AB Audio Amplifier | ECE 521}
%{}
%{\textit{LTspice, Soldering, Oscilloscope}}
%{}
%{}	


\cventry
{Jun 2019}
{ASIC Design | ECE 581}
{}
{\textit{SystemVerilog, Design Compiler, Git, Linux}}
{}
{}
	
	
\cventry
{Sep 2018}
{ARM Sitara AM335x UART / I2C | ECE 372}
{}
{\textit{ARM Assembly, C}}
{}
{}			

\end{document}