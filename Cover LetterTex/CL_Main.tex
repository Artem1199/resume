%% start of file 'template.tex'.
%% Copyright 2006-2013 Xavier Danaux (xdanaux@gmail.com).
%
% This work may be distributed and/or modified under the
% conditions of the LaTeX Project Public License version 1.3c,
% available at http://www.latex-project.org/lppl/.


\documentclass[letterpaper]{moderncv}        % possible options include font size ('10pt', '11pt' and '12pt'), paper size ('a4paper', 'letterpaper', 'a5paper', 'legalpaper', 'executivepaper' and 'landscape') and font family ('sans' and 'roman')
\usepackage{textcomp}
\usepackage{comment}
% moderncv themes
\moderncvstyle{classic}                             % style options are 'casual' (default), 'classic', 'oldstyle' and 'banking'
\moderncvcolor{blue}                               % color options 'blue' (default), 'orange', 'green', 'red', 'purple', 'grey' and 'black'
%\renewcommand{\familydefault}{\sfdefault}         % to set the default font; use '\sfdefault' for the default sans serif font, '\rmdefault' for the default roman one, or any tex font name
%\nopagenumbers{}                                  % uncomment to suppress automatic page numbering for CVs longer than one page

% character encoding
\usepackage[utf8]{inputenc}                       % if you are not using xelatex ou lualatex, replace by the encoding you are using
%\usepackage{CJKutf8}                              % if you need to use CJK to typeset your resume in Chinese, Japanese or Korean

% adjust the page margins
\usepackage[scale=0.85]{geometry}
%\setlength{\hintscolumnwidth}{3cm}                % if you want to change the width of the column with the dates
%\setlength{\makecvtitlenamewidth}{10cm}           % for the 'classic' style, if you want to force the width allocated to your name and avoid line breaks. be careful though, the length is normally calculated to avoid any overlap with your personal info; use this at your own typographical risks...
    % Profile
    
\name{Artem Kulakevich}{}
\address{Beaverton, Oregon}
\phone[mobile]{503-750-3225}
\email{Artem3@pdx.edu}
\homepage{linkedin.com/in/artem-kulakevich/}
\begin{document}

    
%\recipient{XXXXXX}{SOMECOMPANY, Inc.\\123 SOMESTREET\\CITY}
\recipient{ }{ }
\date{\today}
\opening{Hello Akka-Technologies hiring staff,}
\closing{Best Regards,}
%\enclosure[Attached]{curriculum vit\ae{}}          % use an optional argument to use a string other than "Enclosure", or redefine \enclname
\makelettertitle

I have recently graduated from Portland State University's electrical engineering program with summa cum laude honors for achieving a 3.97/4.00 GPA.  I have already started working on my Master's in Electrical Engineering with a focus on digital integrated circuits and embedded systems.  I am nearly halfway done with my Master's degree thanks to my college's accelerated Master’s program.  I have been working full-time throughout the past 4 years while getting my degree and I am only 23 years old!

The majority of my education has been entirely focused on electrical engineering and digital circuits.  I have taken multiple classes on Verilog, SystemVerilog, analog circuit design, digital IC design, and various electronics courses.  I have experience with low-level programming including ARM Assembly, C, C++, and Rust.  I also have some experience with formal verification as well as systems engineering.

My senior capstone involved working with a self-balancing robot to create a high-assurance robot controller.  This involved modifying a program called Kind2 to generate embedded friendly Rust code from a language called Lustre.  I was able to independently modify the code generator, and generate an embedded friendly PID and Fuzzy logic controller that successfully balanced the robot.

I am currently working on a personal project called Centennial which is a self-balancing robot written purely in embedded Rust on an STM32f303 processor.  Rust is a really interesting language and it is fun for me to work on a new language that has not been thoroughly explored.

While going to college, I have also been working at Micro Systems Engineering Inc. for almost four years now.   For work, I have completed multiple official validations, verifications, document updates, and have trained people on those document updates.  I have done LabVIEW programming for live production systems and performed a LabVIEW code review for a senior engineer.  I have worked with Epson 6-axis robots and performed Epson RC+ robot training.  I have hand wired multiple mains voltage robot controllers and I have spent countless hours of troubleshooting electrical and mechanical failures.  I am also required to lead a small team of 8-9 specialists on production tasks a few days out of the week.  I absolutely love the people I work with right now and manage to get along with pretty much everyone, but I believe it is time for me to get a job that fits my education.

\makeletterclosing

\end{document}